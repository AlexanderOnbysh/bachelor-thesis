% зазначаємо стильовий файл, який будемо використовувати

\documentclass{bachelor_thesis}

\usepackage{verbatim}

\newcommand{\bigG}{\mathbb{G}}
\newcommand{\inR}{\in_R}
\newcommand{\Mod}[1]{\ (\text{mod}\ #1)}

\renewcommand{\floatpagefraction}{.8}
\DeclareMathOperator{\ind}{ind}

% починаємо верстку документа
\begin{document}

\maketitlepage{
    StudentName={Oнбиш Олександр Олегович"}, 
    StudentMale=true,
    StudentGroup={ІП-52},
    ThesisTitle={Розпізнавання промовленного тексту по відеоряду міміки людини},
    Advisor={доцент, канд. техн. наук, доцент Гавриленко~О.~В.},
    Reviewer={проф., д-р техн. наук, проф. Сидоренко~С.~С.},
    Year={2016}
}

\assignment{
    StudentName={Онбиш Олександр Олегович},
    StudentMale=true,
    ThesisTitle={Розпізнавання промовленного тексту по відеоряду міміки людини},
    Advisor={Гавриленок О.В., канд. техн. наук, доцент},
    Order={\invcommas{28}~травня~2018~р.~\No~000-C},
    ApplicationDate={\invcommas{15}~червня~2018~р.},
    InputData={розроблювана система повинна працювати з нечіткими даними, мінімальна точність розпізнавання --- 70\%},
    Contents={виконати аналіз існуючих методів розв'язання задачі, вибрати метод розпізнавання промовленноготексту за відеорядом обличчя людини, спроектувати автоматизовану систему розпізнавання емоцій, здійснити програмну реалізацію розробленої системи, провести тестування розробленої системи},
    Graphics={архітектурні графи нейронних мереж, блок-схеми розроблених алгоритмів, схема взаємодії модулів системи, знімки екранних форм},
    ConsultantChapter={Розділ 3. Математичне забезпечення},
    Consultant={Бондаренко~Б.~Б., старший~викладач},
    AssignmentDate={\invcommas{15}~квітня~2018~р.},
    Calendar={
    1 & Огляд літератури за тематикою та збір даних & 12.11.2018 & \\
    \hline
    2 & Проведення порівняльного аналізу математичних методів розпізнавання зображень & 14.12.2018 & \\
    \hline
    3 & Проведення порівняльного аналізу математичних методів розпізнавання емоційного стану за зображенням обличчя & 24.12.2018 & \\
    \hline
    4 & Підготовка матеріалів першого розділу роботи & 01.02.2019 & \\
    \hline
    5 & Розроблення математичного забезпечення для розпізнавання емоційного стану за статичним фронтальним зображенням обличчя & 01.03.2019 & \\
    \hline
    6 & Підготовка матеріалів другого розділу роботи & 15.03.2019 & \\
    \hline
    7 & Підготовка матеріалів третього розділу роботи & 05.04.2019 & \\
    \hline
    8 & Підготовка матеріалів четвертого розділу роботи & 03.05.2019 & \\
    \hline
    9 & Оформлення пояснювальної записки & 01.06.2019 & \\
    },
    StudentNameShort={Онбиш~О.~О.},
    AdvisorNameShort={Гавриленко~О.~В.},
    Year={2019}
}

%!TEX root = ../abstract.tex
% створюємо анотацію
% \setcounter{page}{4}
\chapter*{Реферат}
\pagestyle{empty}
\setfontsize{14}
\thispagestyle{empty}
% далі пишемо текст анотації

% анотація повинна починатися інформацією про структуру роботи
% (кількість аркушів (БЕЗ ДОДАТКІВ!), додатків, посилань, рисунків і таблиць)
Роботу виконано на 42 аркушах, вона містить перелік посилань на використані джерела з $42$ найменувань.

% далі потрібно вказати мету роботи
\textbf{Метою} даної дипломної роботи ииии побудова програмного комплексу для розпізнавання промовленного тексту за відеорядом міміки обляччя людини.

\textbf{Об'єктом дослідження} є something.

\textbf{Предметом дослідження} є something.

Lorem ipsum dolor sit amet, consectetur adipiscing elit. Vivamus malesuada sapien mattis justo pellentesque commodo. Nulla aliquet lorem nec dolor pellentesque, rhoncus vestibulum velit auctor. Suspendisse potenti. Vivamus at sapien velit. Maecenas rhoncus egestas purus sed cursus. Fusce posuere nisl quis sem laoreet faucibus. Ut cursus at libero et iaculis. Donec finibus, nunc sit amet cursus lobortis, turpis lectus consectetur neque, non accumsan felis diam eget lorem. Mauris auctor dolor arcu, a aliquam diam vestibulum ut. Maecenas a cursus lectus, egestas eleifend magna. 
 % наприкінці анотації потрібно зазначити ключові слова
\MakeUppercase{Lorem, ipsum, dolor, sit, amet}

% створюємо анотацію англійською мовою
\chapter*{Abstract}
\thispagestyle{empty}
% анотація повинна починатися інформацією про структуру роботи
% (кількість аркушів (БЕЗ ДОДАТКІВ!), додатків, посилань, рисунків і таблиць)
The thesis is presented in N pages. It contains bibliography of N references.

% % далі потрібно вказати мету роботи

The \textbf{goal} Lorem ipsum dolor sit amet, consectetur

\textbf{The object} Lorem ipsum dolor sit amet, consectetur

\textbf{The subject} Lorem ipsum dolor sit amet, consectetur


% % далі потрібно вказати розглянуті методи та критерії, за яким вибрано один із них

Lorem ipsum dolor sit amet, consectetur adipiscing elit. Vivamus malesuada sapien mattis justo pellentesque commodo. Nulla aliquet lorem nec dolor pellentesque, rhoncus vestibulum velit auctor. Suspendisse potenti. Vivamus at sapien velit. Maecenas rhoncus egestas purus sed cursus. Fusce posuere nisl quis sem laoreet faucibus. Ut cursus at libero et iaculis. Donec finibus, nunc sit amet cursus lobortis, turpis lectus consectetur neque, non accumsan felis diam eget lorem. Mauris auctor dolor arcu, a aliquam diam vestibulum ut. Maecenas a cursus lectus, egestas eleifend magna. 
 
% % далі потрібно коротко викласти суть роботи
% The selected  is implemented in parallel computation model and is executed on the selected  platform. From the results of this execution analysis of optimal algorithm parameters and approximate problem size solvable in the span of one calendar year were made.

% % далі потрібно подати відомості про апробацію роботи


% The results could be used for estimating attack cost on popular asymmetric .

% % наприкінці анотації потрібно зазначити ключові слова
\MakeUppercase{Lorem, ipsum, dolor, sit, amet}

% створюємо зміст
% \includepdf[pages={-}]{abstract.pdf}
\pagenumbering{gobble}
\tableofcontents
\cleardoublepage
\pagenumbering{arabic}
\setcounter{page}{8}

% створюємо перелік умовних позначень, скорочень і термінів
%!TEX root = ../thesis.tex
% створюємо перелік умовних позначень, скорочень і термінів
\shortings

\textbf{Back propagation} --- алгоритм зворотнього поширення помилки  

\textbf{CNN} --- convolutional neural network - згорткова нейронна мережа

Automatic speech recogintion ASR
VGG
image features
video features
optical flow
CTC loss
loss function
%!TEX root = ../thesis.tex
% створюємо вступ
 \intro

Читання по губам відіграє вирішальну роль у людському спілкуванні і розумінні мови, цей факт підтверджується багатьма дослідженнями, одне з яких є дослідження МакГурка (McGurk \& MacDonald, 1976). У своїй роботі МакГурк вивчав еффект коли поверх відео с обличчям людини яка промовляє певні фонеми накладається звуковий ряд на якому звучать зовсім інші фонеми, таким чином людина сприймає третій вид фонеми яка відрізняється від тої яка буда на відеоряді та аудіозаписі. Це дослідження підтверджує що людина сприймає не лише звукові сигнали але й співставляє їх с візуальною інформацією яка включає в себе міміку людини та її артикуляцію. 

Читання по губам є важкою задачею для людей, особливо у відсутності контексту. Більшість положень облицця, губ, зубів та іноді язика є латентними таким чином утрорюючи складність для розпізнавання без додаткового контексту \cite{doi:10.1044/jshr.1104.796}. Наприклад Фішер (1968) дає 5 категорій візуальних фонем (візем), зі списку 23 базових фонем, які зазвичай плутаються людьми коли вони спостерігають за артикуляюєю розмовника. Таким чином люди погано справляються з задачею чинання по губам. 

Люди з вадами слуху досягають лише $17 \pm 11\%$ точності для обмедженної підмножини з $30$ односкладових слів та $21 \pm 11\%$ для 30 складених слів \cite{easton1982perceptual}. Таким чином задача автоматизації читання по губам є дуже важливою. Автоматизація читання по губам має безліч практичних застосувань: 

\begin{itemize}
    \item допомога людям з вадами слуху
    \item тихе диктування у публічних місцях
    \item безпека
    \item розпізнавання промовленного тексту у шумних місцях
    \item біометрична ідентифікація людини
\end{itemize}

Загальна процедура читання губ включає два етапи: аналіз відео інформації міміки обличчя у відеоряді, перетворення цієї інформації в слова або речення. Ця процедура пов'язує читання по губам з двома близькими напрямками: розпізнавання слів на основі аудіо записів та розпізнавання рукописного тексту яких спирається на аналогічний аналіз вхідних послідовностей. Проте сьогодні існує великий розрив у точності між читанням по губам і цими двома тісно пов'язані задачами. Однією з головних причин є  складність задачі.

З розвитком машинного навчання а зокрема напрямку компьютерного зору насьогодні зявляється можливість вирішення данної задачі за допомогою глибокого навчання згорткових нейронних мереж.
\chapter{АНАЛІЗ ІСНУЮЧИХ ПРОГРАМНИХ РЕШЕНЬ}

У данному розділі описуються різноманітні підходи для автоматизації задачі читання
по губам.

\section{Розпізнавання слів класичними методами машинного навчання} 
Більшість сучасних методів для розпізнавання промовленного тексту не використовує технологій глубинного навчання. Такий підхід потребує дуже багато обчислень на стадії підготовки данних. Одні з основних предобчислень є:

\begin{itemize}
    \item image embeddings
    \item video embeddings
    \item optical flows
\end{itemize}

Такий підхід включає в себе багато алгоритмізованих підходів які є результатом спостережень та експериментів і він не є еффективним ані з точки зору швидкості обчислень ані з точки зору точності. 

\cite{goldschen1997continuous} було першою спробою зробити автоматизонае читання по губам на рівні речень. У цій імплементації використовано приховані марковські моделі (hidden Markov models) як модель для розпізнавання. 

\cite{neti2000audio} першими зробили аудіовізуальне розпізнавання мови у речення за допомогою прихованих марковські моделі (hidden Markov models) та власноруч побудованих дескриптораху, на наборі даних IBM ViaVoice. Автори покращують продуктивність розпізнавання мовлення в шумному оточенні шляхом злиття візуальних інформації із звуковою. Набір даних виступаючий для навчання містить $17111$ висловлень від $261$ людей (близько $34,9$ годин відео) і не є загальнодоступним. 

Автори зазначають що їхні візуальні результати не можуть бути інтерпретовані як тільки візуальне розпізнавання, оскільки вони використовували аудіодоріжку для підкріплення рішень своєї моделі. Використовуючи модель на основі прихованої марковської моделі автори отримали результати $ 91,62\%$ точності для моделі яка вивчається окремо для кожної людини, та $ 82,31 \%$  для загальної модели. Для перевірки було використано набір данних WER. Також ця модель показує $ 38,53 \%$, $ 16,77 \%$ на наборі данних WER з'єднаним з корпусом DIGIT, який містить промовленні цифри. 

Крім того такий підхід змушує адаптувати модель для кожної людини на якій він буде використовуватися. Генералізація модели для розпізнавання промовленного тексту незалежно від особи яка його промовляє залишаєтся невирішоную задачею. 

\section{Розпізнавання слів за допомогою глибокого навчання}

В останні роки було зроблено декілька спроб застосувати глибоке навчання для задачі розпізнавання промовленного тексту. Проте всі ці підходи виконують лише класифікацію слова або фонеми, і жоден з методів не робить повне передбачення послідовності речень. Данні підходи включають вивчення мультимодальних аудіовізуальних представлень, \cite{ngiam2011multimodal, sui2015listening, petridis2016deep}. недоліком цих робіт є те, що вони вокористовують традиційні підходи для класифікації слів та / або фонем які використовувалися виключно для аудіо обробки (наприклад, HMMs, GMM-HMMs і т.д.) \cite{almajai:2016, takashima2016audio, noda2014lipreading, koller2015deep}.

\cite{chung2016lip} пропонує просторові та просторово-часові згорткові нейронні мережі на основі VGG для класифікації слів. Архітектури перевірялася на наборі даних на рівні слова BBC TV (333 і 500 класів), але, як повідомлялося, їх просторово-часові моделі уступають просторовим архітектурам в середньому на $ 14 \$$. Крім того, їхні моделі не можуть обробляти змінні довжини послідовностей, і вони не намагаються передбачати послідовність на рівні речення. 

\cite{chung2016out} тренують аудіовізуальну модель для классифікації використовуючи вже натренеровані моделі для виділення image features з зображення обличчя людини. Отриманні фічі подають на вход до рекурентної нейроної мережі LSTM. Модель була перевірена на наборі данних OuluVS2.

\cite{wand2016lipreading} представляють модель основану на рекурентних неронних мережах, а зокрема LSTM. Недоліком данного рішення є те, що передбачення моделі базується не на реченнях, а на словах, також ця модель не є незалежною від особи, тому має бути дотренерована на окремих людях.

\cite{garglip} застосовують преднавчену на обличчях згорткову нейронну мережу VGG для классифікації слів та речень на наборі данних MIRACL-VC1. Недоліком цього дослідження є те, що набір данних включає в себе лише 10 слів та 10 реччень. Ще одним недоліком є те, що згорткова нейронна мережа VVG була навченою та використовувалася лише як feature extractor а усі передбачення вивчалися через рекурентну мережу на освнові LSTM, таким чином модель навчалася у два етапи а не в один. Найкраща модель мала $56.0\%$ точності на задачі классифікації слів та $44.5\%$ на задачі классифікації речень.

\section{Розпізнавання послідовності слів за допомогою глибокого навчання}

Напрямок автоматичного розпізнавання промовленого тексту (ASR) не мав би такого потенціалу без сучасних досліджень в області машинного навчання а точніше в області глибоких нейронних мереж, багато з яких були зроблені в котнексті ASR \cite{graves2006connectionist,dahl2012context,hinton2012deep}. Також суттевим є внесок розробки специфічних функій втрат (loss function) одною з яких є CTC loss, головною метою якої є вирішення проблеми які виникають при навчанні моделей на послідовностях данних таких як рукописний текст, аудіо зоапис, або як у випадку роспізнавання промовлених слів відео запис. 


\begin{comment}

LipNet is the first end-to-end model that performs sentence-level sequence prediction for visual speech recogntion. That is, we 
demonstrate the first work that takes as input as sequence of images and outputs a distribution over sequences of tokens; it is 
trained end-to-end using CTC and thus also does not require alignments.
\end{comment}

\chapter{Обґрунтування вибору засобів реалізації}

У цьому розділі дано обґгунтування вибору мови програмування, бібліотек для навчання нейронних мереж та фреймворків для реалізації серверної та клієнтської частин. 

\section{Вибір мови програмування для розроблення серверної частини}
Вимоги до мови программування, що випливають з постановки задачі:

\begin{enumerate}
    \item Наявність бібліотек для розробки неронних мереж з підтримкою CUDA
    \item Наявність бібліотек для обробки відео та зображень
    \item Наявність фреймворків для HTTP веб застосувань
    \item Швидкодія
\end{enumerate}


\subsection{Python}

Python дуже популярна і широко використовується мова програмування. Це мова програмування загального призначення. Python пропонує систему динамічних типів і автоматизоване управління пам'яттю. Python підтримує багато парадигм програмування, таких як: об'єктно-орієнтоване, функціональне, імперативне, процедурне. Однією з переваг цієї мови програмування є велика стандартна бібліотека. Ці бібліотеки широко використовуються в промисловості і добре зарекомендували себе.

\begin{enumerate}
\item Мова програмування Python має вилику кількисть бібліотек для обробки данних та манинного вавчання, наприклад, NumPy - бібліотека для роботі в числовими масивами данних, використовується для роботи з векторами та матрицями. Бібліотека Scikit-learn надає інструменти для інтелектуального аналізу даних. Додаток Pandas надає розробникам високопродуктивні структури данних та інструменти для їх аналізу. Аналогічно, SciPy використовується для математичних обчислень. Також для мови Python має підтримку великої кількості бібліотке для тренування нейронних мереж таких як Pytorch, Tensorflow, Caffe, MXNet.
\item Мова Python має декілько бібліотке для обробки зображень та відеофайлів. OpenCV (Open Source Computer Vision Library) відкрита бібліотека для роботи над алгоритмами компьютерного зору та машинного навчання. Бібліотека має більше ніж 2500 оптимізованих алгоритмів, що включає в себе повний набір як класичних, так і найсучасніших алгоритмів комп'ютерного зору і машинного навчання. Ці алгоритми можуть бути використані для виявлення та розпізнавання облич, визначення об'єктів, класифікації дій людини у відео, відстеження рухів камери, відстеження рухомих об'єктів, вилучення 3D-моделей об'єктів, створення 3D-хмари точок від стереокамер, зшивання зображень для отримання високого дозволу зображення всієї сцени, пошук схожих зображень з бази даних зображень.
\item 
\end{enumerate}



% \section{Вибір технологій для розроблення серверної частини}

% test

% \section{Вибір технологій для розроблення нейронної мережі}

% test

% \section{Вибір технологій для розроблення клієнтської частини}
\input{chapters/6_algorithms}
\input{chapters/7_implementation}
%!TEX root = ../thesis.tex
% створюємо Висновки до всієї роботи
\conclusions

Lorem ipsum dolor sit amet, consectetur adipiscing elit. Vivamus malesuada sapien mattis justo pellentesque commodo. 
Nulla aliquet lorem nec dolor pellentesque, rhoncus vestibulum velit auctor. Suspendisse potenti. Vivamus at sapien velit.
 Maecenas rhoncus egestas purus sed cursus. Fusce posuere nisl quis sem laoreet faucibus. Ut cursus at libero et iaculis. 
 Donec finibus, nunc sit amet cursus lobortis, turpis lectus consectetur neque, non accumsan felis diam eget lorem. Mauris 
 auctor dolor arcu, a aliquam diam vestibulum ut. Maecenas a cursus lectus, egestas eleifend magna.


Доцільними напрямками подальшої роботи можуть бути:

\begin{itemize}
    \item Lorem ipsum dolor sit amet, consectetur
    \item Lorem ipsum dolor sit amet, consectetur
    \item Lorem ipsum dolor sit amet, consectetur
\end{itemize}


\bibliographystyle{ugost2003}
\bibliography{thesis}


% % створюємо додатки
% % перший додаток повинен містити лістинги розроблених програм
% \append{Лістинги програм}

% % кожний лістинг вставляється в додаток за допомогою спеціальної команди,
% % перший аргумент якої --- це заголовок, який з'являтиметься в тексті,
% % другий --- шлях до файлу з лістингом
% \listing{presenter.h --- прототип пред'явника-вчителя}{SRC/Presenter/presenter.h}

% \listing{presenter.cpp --- реалізація пред'явника-вчителя}{SRC/Presenter/presenter.cpp}

% \input{chapters/appendix_firstrun}

% % останній додаток повинен містити слайди пр\езентації доповіді на захисті дипломної роботи
% \append{Ілюстративний матеріал}

% \begin{figure}[!htp]%
% 	\centering
% 	\includegraphics[scale = 0.43]{PNG/slide1.png}%
% 	\caption{Слайд 1}%
% 	\label{fig:p1}%
% \end{figure}

% \begin{figure}[!htp]%
% 	\centering
% 	\includegraphics[scale = 0.455]{PNG/slide2.png}%
% 	\caption{Слайд 2}%
% 	\label{fig:p2}%
% \end{figure}

\end{document}
