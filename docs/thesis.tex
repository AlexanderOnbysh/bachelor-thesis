% зазначаємо стильовий файл, який будемо використовувати

\documentclass{bachelor_thesis}
% \usepackage{fontspec}

% \usepackage{mathtools}

\usepackage{verbatim}
\usepackage{wrapfig}
\usepackage{bm}
% \usepackage{pdfpages}

\newcommand{\bigG}{\mathbb{G}}
\newcommand{\inR}{\in_R}
\newcommand{\Mod}[1]{\ (\text{mod}\ #1)}

\renewcommand{\floatpagefraction}{.8}
\DeclareMathOperator{\ind}{ind}


% починаємо верстку документа
\begin{document}

%!TEX root = ../abstract.tex
% створюємо анотацію
% \setcounter{page}{4}
\chapter*{Реферат}
\pagestyle{empty}
\setfontsize{14}
\thispagestyle{empty}
% далі пишемо текст анотації

% анотація повинна починатися інформацією про структуру роботи
% (кількість аркушів (БЕЗ ДОДАТКІВ!), додатків, посилань, рисунків і таблиць)
Роботу виконано на N аркушах, вона містить перелік посилань на використані джерела з $N^2$ найменувань.

% далі потрібно вказати мету роботи
\textbf{Метою} даної дипломної роботи є побудова something.

\textbf{Об'єктом дослідження} є something.

\textbf{Предметом дослідження} є something.

Lorem ipsum dolor sit amet, consectetur adipiscing elit. Vivamus malesuada sapien mattis justo pellentesque commodo. 
Nulla aliquet lorem nec dolor pellentesque, rhoncus vestibulum velit auctor. Suspendisse potenti. Vivamus at sapien velit.
 Maecenas rhoncus egestas purus sed cursus. Fusce posuere nisl quis sem laoreet faucibus. Ut cursus at libero et iaculis. 
 Donec finibus, nunc sit amet cursus lobortis, turpis lectus consectetur neque, non accumsan felis diam eget lorem. Mauris 
 auctor dolor arcu, a aliquam diam vestibulum ut. Maecenas a cursus lectus, egestas eleifend magna.
 
 % наприкінці анотації потрібно зазначити ключові слова
\MakeUppercase{Lorem, ipsum, dolor, sit, amet}

% створюємо анотацію англійською мовою
\chapter*{Abstract}
\thispagestyle{empty}
% анотація повинна починатися інформацією про структуру роботи
% (кількість аркушів (БЕЗ ДОДАТКІВ!), додатків, посилань, рисунків і таблиць)
The thesis is presented in N pages. It contains bibliography of N references.

% % далі потрібно вказати мету роботи

The \textbf{goal} Lorem ipsum dolor sit amet, consectetur

\textbf{The object} Lorem ipsum dolor sit amet, consectetur

\textbf{The subject} Lorem ipsum dolor sit amet, consectetur


% % далі потрібно вказати розглянуті методи та критерії, за яким вибрано один із них

Lorem ipsum dolor sit amet, consectetur adipiscing elit. Vivamus malesuada sapien mattis justo pellentesque commodo. 
Nulla aliquet lorem nec dolor pellentesque, rhoncus vestibulum velit auctor. Suspendisse potenti. Vivamus at sapien velit.
 Maecenas rhoncus egestas purus sed cursus. Fusce posuere nisl quis sem laoreet faucibus. Ut cursus at libero et iaculis. 
 Donec finibus, nunc sit amet cursus lobortis, turpis lectus consectetur neque, non accumsan felis diam eget lorem. Mauris 
 auctor dolor arcu, a aliquam diam vestibulum ut. Maecenas a cursus lectus, egestas eleifend magna.
 
 
% % далі потрібно коротко викласти суть роботи
% The selected  is implemented in parallel computation model and is executed on the selected  platform. From the results of this execution analysis of optimal algorithm parameters and approximate problem size solvable in the span of one calendar year were made.

% % далі потрібно подати відомості про апробацію роботи


% The results could be used for estimating attack cost on popular asymmetric .

% % наприкінці анотації потрібно зазначити ключові слова
\MakeUppercase{Lorem, ipsum, dolor, sit, amet}

% створюємо зміст
% \includepdf[pages={-}]{abstract.pdf}
\pagenumbering{gobble}
\tableofcontents
\cleardoublepage
\pagenumbering{arabic}
\setcounter{page}{8}

% створюємо перелік умовних позначень, скорочень і термінів

%!TEX root = ../thesis.tex
% створюємо перелік умовних позначень, скорочень і термінів
\shortings
%!TEX root = ../thesis.tex
% створюємо вступ
\intro
\pagestyle{plain}

\textbf{Актуальність роботи.} 


\textbf{Метою роботи} \cite{1512.03385}

У ході дослідження ставляться наступні \textbf{завдання}:

\begin{itemize}
    \item some

\end{itemize}

\textbf{Методи дослідження}: 

\emph{Об’єкт дослідження}: 

\emph{Предмет дослiдження}: 

\textbf{test

\textbf{test

% висновки
%!TEX root = ../thesis.tex
% створюємо Висновки до всієї роботи
\conclusions

Lorem ipsum dolor sit amet, consectetur adipiscing elit. Vivamus malesuada sapien mattis justo pellentesque commodo. 
Nulla aliquet lorem nec dolor pellentesque, rhoncus vestibulum velit auctor. Suspendisse potenti. Vivamus at sapien velit.
 Maecenas rhoncus egestas purus sed cursus. Fusce posuere nisl quis sem laoreet faucibus. Ut cursus at libero et iaculis. 
 Donec finibus, nunc sit amet cursus lobortis, turpis lectus consectetur neque, non accumsan felis diam eget lorem. Mauris 
 auctor dolor arcu, a aliquam diam vestibulum ut. Maecenas a cursus lectus, egestas eleifend magna.


Доцільними напрямками подальшої роботи можуть бути:

\begin{itemize}
    \item Lorem ipsum dolor sit amet, consectetur
    \item Lorem ipsum dolor sit amet, consectetur
    \item Lorem ipsum dolor sit amet, consectetur
\end{itemize}


\bibliographystyle{ugost2003}
\bibliography{thesis}
% % створюємо додатки
% % перший додаток повинен містити лістинги розроблених програм
% \append{Лістинги програм}

% % кожний лістинг вставляється в додаток за допомогою спеціальної команди,
% % перший аргумент якої --- це заголовок, який з'являтиметься в тексті,
% % другий --- шлях до файлу з лістингом
% \listing{presenter.h --- прототип пред'явника-вчителя}{SRC/Presenter/presenter.h}

% \listing{presenter.cpp --- реалізація пред'явника-вчителя}{SRC/Presenter/presenter.cpp}

% \input{chapters/appendix_firstrun}

% % останній додаток повинен містити слайди пр\езентації доповіді на захисті дипломної роботи
% \append{Ілюстративний матеріал}

% \begin{figure}[!htp]%
% 	\centering
% 	\includegraphics[scale = 0.43]{PNG/slide1.png}%
% 	\caption{Слайд 1}%
% 	\label{fig:p1}%
% \end{figure}

% \begin{figure}[!htp]%
% 	\centering
% 	\includegraphics[scale = 0.455]{PNG/slide2.png}%
% 	\caption{Слайд 2}%
% 	\label{fig:p2}%
% \end{figure}

\end{document}
