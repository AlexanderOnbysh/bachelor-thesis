% зазначаємо стильовий файл, який будемо використовувати

\documentclass{bachelor_thesis}

% \usepackage{mathtools}

\usepackage{verbatim}
\usepackage{wrapfig}
\usepackage{bm}
\usepackage{scrpage2}
% \usepackage{pdfpages}

\newcommand{\bigG}{\mathbb{G}}
\newcommand{\inR}{\in_R}
\newcommand{\Mod}[1]{\ (\text{mod}\ #1)}

\renewcommand{\floatpagefraction}{.8}
\DeclareMathOperator{\ind}{ind}

% нумерация в нижнем правом углу
\ifoot[]{}
\cfoot[]{}
\ofoot[\pagemark]{\pagemark}
\pagestyle{scrplain}

% починаємо верстку документа
\begin{document}

\input{chapters/abstract}

% створюємо зміст
% \includepdf[pages={-}]{abstract.pdf}
\pagenumbering{gobble}
\tableofcontents
\cleardoublepage
\pagenumbering{arabic}
\setcounter{page}{8}

% створюємо перелік умовних позначень, скорочень і термінів

%!TEX root = ../thesis.tex
% створюємо перелік умовних позначень, скорочень і термінів
\shortings

\textbf{Back propagation} --- алгоритм зворотнього поширення помилки  

\textbf{CNN} --- convolutional neural network - згорткова нейронна мережа
%!TEX root = ../thesis.tex
% створюємо вступ
\intro
\pagestyle{plain}

\textbf{Актуальність роботи.} 


\textbf{Метою роботи} \cite{1512.03385}

У ході дослідження ставляться наступні \textbf{завдання}:

\begin{itemize}
    \item some

\end{itemize}

\textbf{Методи дослідження}: 

\emph{Об’єкт дослідження}: 

\emph{Предмет дослiдження}: 

\textbf{test

\textbf{test
\input{chapters/overview}

% висновки
\input{chapters/conclusions}

\bibliographystyle{ugost2003}
\bibliography{thesis}
% % створюємо додатки
% % перший додаток повинен містити лістинги розроблених програм
% \append{Лістинги програм}

% % кожний лістинг вставляється в додаток за допомогою спеціальної команди,
% % перший аргумент якої --- це заголовок, який з'являтиметься в тексті,
% % другий --- шлях до файлу з лістингом
% \listing{presenter.h --- прототип пред'явника-вчителя}{SRC/Presenter/presenter.h}

% \listing{presenter.cpp --- реалізація пред'явника-вчителя}{SRC/Presenter/presenter.cpp}

% \input{chapters/appendix_firstrun}

% % останній додаток повинен містити слайди пр\езентації доповіді на захисті дипломної роботи
% \append{Ілюстративний матеріал}

% \begin{figure}[!htp]%
% 	\centering
% 	\includegraphics[scale = 0.43]{PNG/slide1.png}%
% 	\caption{Слайд 1}%
% 	\label{fig:p1}%
% \end{figure}

% \begin{figure}[!htp]%
% 	\centering
% 	\includegraphics[scale = 0.455]{PNG/slide2.png}%
% 	\caption{Слайд 2}%
% 	\label{fig:p2}%
% \end{figure}

\end{document}
