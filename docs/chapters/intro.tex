%!TEX root = ../thesis.tex
% створюємо вступ
    \intro

Читання по губам відіграє вирішальну роль у людському спілкуванні і розумінні мови, цей факт 
підтверджується багатьма дослідженнями, одне з яких є дослідження МакГурка (McGurk \& MacDonald, 1976). 
У своїй роботі МакГурк вивчав еффект коли поверх відео с обличчям людини яка промовляє певні фонеми 
накладається звуковий ряд на якому звучать зовсім інші фонеми, таким чином людина сприймає третій вид 
фонеми яка відрізняється від тої яка буда на відеоряді та аудіозаписі. Це дослідження підтверджує що 
людина сприймає не лише звукові сигнали але й співставляє їх с візуальною інформацією яка включає в 
себе міміку людини та її артикуляцію.

Читання по губам є важкою задачею для людей, особливо у відсутності контексту. Більшість положень облицця, 
губ, зубів та іноді язика є латентними таким чином утрорюючи складність для розпізнавання без додаткового 
контексту \cite{doi:10.1044/jshr.1104.796}. Наприклад Фішер (1968) дає 5 категорій візуальних фонем (візем), зі 
списку 23 базових фонем, які зазвичай плутаються людьми коли вони спостерігають за артикуляюєю розмовника. 
Таким чином люди погано справляються з задачею чинання по губам. 

Люди з вадами слуху досягають лише $17 \pm 11\%$ точності для обмедженної підмножини з $30$ односкладових слів 
та $21 \pm 11\%$ для 30 складених слів \cite{easton1982perceptual}. Таким чином задача автоматизації читання по 
губам є дуже важливою. Автоматизація читання по губам має безліч практичних застосувань:

\begin{itemize}
    \item допомога людям з вадами слуху
    \item тихе диктування у публічних місцях
    \item безпека
    \item розпізнавання промовленного тексту у шумних місцях
    \item біометрична ідентифікація людини
\end{itemize}

Загальна процедура читання губ включає два етапи: аналіз відео інформації міміки обличчя у відеоряді, 
перетворення цієї інформації в слова або речення. Ця процедура пов'язує читання по губам з двома близькими 
напрямками: розпізнавання слів на основі аудіо записів та розпізнавання рукописного тексту яких 
спирається на аналогічний аналіз вхідних послідовностей. Проте сьогодні існує великий розрив у 
точності між читанням по губам і цими двома тісно пов'язані задачами. Однією з головних причин 
є  складність задачі.

З розвитком машинного навчання а зокрема напрямку компьютерного зору насьогодні зявляється можливість 
вирішення данної задачі за допомогою глибокого навчання згорткових нейронних мереж. 