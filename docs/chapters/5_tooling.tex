
\chapter{Обґрунтування вибору засобів реалізації}

У цьому розділі дано обґгунтування вибору мови програмування, бібліотек для навчання нейронних мереж та фреймворків для реалізації серверної та клієнтської частин. 

\section{Вибір мови програмування для розроблення серверної частини}
Вимоги до мови программування, що випливають з постановки задачі:

\begin{enumerate}
    \item Наявність бібліотек для розробки неронних мереж з підтримкою CUDA
    \item Наявність бібліотек для обробки відео та зображень
    \item Наявність фреймворків для HTTP веб застосувань
    \item Швидкодія
\end{enumerate}


\subsection{Python}

Python дуже популярна і широко використовується мова програмування. Це мова програмування загального призначення. Python пропонує систему динамічних типів і автоматизоване управління пам'яттю. Python підтримує багато парадигм програмування, таких як: об'єктно-орієнтоване, функціональне, імперативне, процедурне. Однією з переваг цієї мови програмування є велика стандартна бібліотека. Ці бібліотеки широко використовуються в промисловості і добре зарекомендували себе.

\begin{enumerate}
\item Мова програмування Python має вилику кількисть бібліотек для обробки данних та манинного вавчання, наприклад, NumPy - бібліотека для роботі в числовими масивами данних, використовується для роботи з векторами та матрицями. Бібліотека Scikit-learn надає інструменти для інтелектуального аналізу даних. Додаток Pandas надає розробникам високопродуктивні структури данних та інструменти для їх аналізу. Аналогічно, SciPy використовується для математичних обчислень. Також для мови Python має підтримку великої кількості бібліотке для тренування нейронних мереж таких як Pytorch, Tensorflow, Caffe, MXNet.
\item Мова Python має декілько бібліотке для обробки зображень та відеофайлів. OpenCV (Open Source Computer Vision Library) відкрита бібліотека для роботи над алгоритмами компьютерного зору та машинного навчання. Бібліотека має більше ніж 2500 оптимізованих алгоритмів, що включає в себе повний набір як класичних, так і найсучасніших алгоритмів комп'ютерного зору і машинного навчання. Ці алгоритми можуть бути використані для виявлення та розпізнавання облич, визначення об'єктів, класифікації дій людини у відео, відстеження рухів камери, відстеження рухомих об'єктів, вилучення 3D-моделей об'єктів, створення 3D-хмари точок від стереокамер, зшивання зображень для отримання високого дозволу зображення всієї сцени, пошук схожих зображень з бази даних зображень.
\item Одною із сильних сторін мови программування Python є існування великої кількості бібліотек для створення WEB застосувань. Одними з самих популярних є Aiohttp, Fask, Django та CherryPy.
\item Python інтерпритована мова программування, тобто код на Python виконується рядок за рядком. Таким чином, Python часто призводить до повільного виконання програми порівняно з іншими мовами програмування. Але завдяки тому що більша частина бібліотек використовує типізовану мову Cython для виконання важких обчислень падіння швидкодії є невеликою.
\end{enumerate}

\subsection{JavaScript}
Javascript - це мультипарадигміна мова програмування. JavaScript підтримує об'єктно-орієнтований, імперативний та функціональний стилі програмування. Найбільш широке застосування мова JavaScript знайшла в браузерах як мова сценаріїв для надання інтерактивності веб-сторінкам. Основні архітектурні особливості: динамічна типізація, слабка типізація, автоматичне управління пам'яттю, функції як об'єкти першого класу. JavasSript не має багато бібліотек для машинного навчання, але для створення UI програми можна використовувати Javascript. Усі бібліотеки доступні через NPM (менеджер пакетів Javascript)
\begin{enumerate}
    \item Бібліотека Tensorflow.js у 2019 році стала дуже популярною для машинного навчання Javascript завдяки його ядра для лінійної алгебри та підтримки . Вона швидко наздогнала свою сестру Python в кількості підтримуваних API і майже будь-які проблеми в машинному навчанні можуть бути вирішені за допомогою неї на цьому етапі.
    ml.js
    Ця бібліотека являє собою набір інструментів, розроблених організацією mljs. Вона включає в себе величезний список бібліотек під різними категоріями, таких як навчання без нагляду, навчання під наглядом, штучні нейронні мережі, регресія, оптимізація, статистика, обробка даних та математичні утиліти.
    Більшість бібліотек, що входять до ml.js, зазвичай використовуються у веб-браузері, але якщо ви хочете працювати з ними в середовищі Node.js, ви знайдете пакет npm.
    \item OpenCV.js - це прив'язка JavaScript для обраної підсистеми функцій OpenCV для веб-платформи. Це дозволяє новим веб-додаткам з мультимедійною обробкою скористатися широким спектром функцій бачення, доступних у OpenCV. OpenCV.js використовує Emscripten для компіляції функцій OpenCV в цілі asm.js або WebAssembly, а також надає JavaScript API для доступу до веб-додатків. У майбутніх версіях бібліотеки скористаються API прискорення, які доступні в Інтернеті, такі як SIMD і багатопотокове виконання.
    \item Метою розробки JavaScript було спростити розродку WEB застосувань, одною з найпопулярніших розробок у цій області є NodeJs.
    \item Node.js - крос-платформни JavaScript середовищем виконання, яке виконує код JavaScript поза браузером. Node.js дозволяє розробникам використовувати JavaScript для написання інструментів командного рядка і для сценаріїв на стороні сервера - запуск скриптів на стороні сервера для створення динамічного вмісту веб-сторінки до того, як сторінка буде відправлена до веб-браузера користувача. Отже, Node.js представляє парадигму "скрізь JavaScript" [7], що об'єднує розробку веб-додатків навколо однієї мови програмування, а не на різних мовах для серверних і клієнтських скриптів.
\end{enumerate}

\subsection{R}
 R - це мова і середовище для статистичних обчислень і графіки. R надає широкий спектр статистичних (лінійних та нелінійних моделей, класичних статистичних тестів, аналізу часових рядів, класифікації, кластеризації) і графічних методів і є дуже розширюваним. Мова S часто є засобом вибору для досліджень у статистичній методології, а R забезпечує шлях відкритого джерела для участі в цій діяльності.
\begin{enumerate}
    \item pass
    \item pass
    \item pass
\end{enumerate}


\begin{table}{| c | c | c | c |}{Порівняння мов програмування}{table:comp32}
    \hline
    {Вимога / Мова програмування} & {Python} & {JavaScript} & {R} \\
    \hline
    \makecell{Наявність бібліотек для розробки \\ неронних мереж з підтримкою CUDA} & 5 & 3 & 2 \\
    \hline
    \makecell{Наявність бібліотек для \\ обробки відео та зображень} & 4 & 3 & 3 \\
    \hline
    \makecell{Наявність фреймворків для HTTP \\ веб застосувань} & 4 & 5 & 2 \\
    \hline
    \makecell{Швидкодія} & 3 & 4 & 4 \\
    \hline
    \makecell{Висновок} & 16 & 15 & 11 \\
 \end{table}


\section{Вибір фреймворку для розробки нейронної мережі}



% test

% \section{Вибір технологій для розроблення клієнтської частини}