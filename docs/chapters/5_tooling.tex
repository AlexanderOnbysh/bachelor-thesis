
\chapter{Обґрунтування вибору засобів реалізації}

У цьому розділі дано обґгунтування вибору мови програмування, бібліотек для навчання нейронних мереж та фреймворків для реалізації серверної та клієнтської частин. 

\section{Вибір мови програмування для розроблення серверної частини}
Вимоги до мови программування, що випливають з постановки задачі:

\begin{enumerate}
    \item Наявність бібліотек для розробки неронних мереж з підтримкою CUDA
    \item Наявність бібліотек для обробки відео та зображень
    \item Наявність фреймворків для HTTP веб застосувань
    \item Швидкодія
\end{enumerate}


\subsection{Python}

Python дуже популярна і широко використовується мова програмування. Це мова програмування загального призначення. Python пропонує систему динамічних типів і автоматизоване управління пам'яттю. Python підтримує багато парадигм програмування, таких як: об'єктно-орієнтоване, функціональне, імперативне, процедурне. Однією з переваг цієї мови програмування є велика стандартна бібліотека. Ці бібліотеки широко використовуються в промисловості і добре зарекомендували себе.

\begin{enumerate}
\item Мова програмування Python має вилику кількисть бібліотек для обробки данних та манинного вавчання, наприклад, NumPy - бібліотека для роботі в числовими масивами данних, використовується для роботи з векторами та матрицями. Бібліотека Scikit-learn надає інструменти для інтелектуального аналізу даних. Додаток Pandas надає розробникам високопродуктивні структури данних та інструменти для їх аналізу. Аналогічно, SciPy використовується для математичних обчислень. Також для мови Python має підтримку великої кількості бібліотке для тренування нейронних мереж таких як Pytorch, Tensorflow, Caffe, MXNet.
\item Мова Python має декілько бібліотке для обробки зображень та відеофайлів. OpenCV (Open Source Computer Vision Library) відкрита бібліотека для роботи над алгоритмами компьютерного зору та машинного навчання. Бібліотека має більше ніж 2500 оптимізованих алгоритмів, що включає в себе повний набір як класичних, так і найсучасніших алгоритмів комп'ютерного зору і машинного навчання. Ці алгоритми можуть бути використані для виявлення та розпізнавання облич, визначення об'єктів, класифікації дій людини у відео, відстеження рухів камери, відстеження рухомих об'єктів, вилучення 3D-моделей об'єктів, створення 3D-хмари точок від стереокамер, зшивання зображень для отримання високого дозволу зображення всієї сцени, пошук схожих зображень з бази даних зображень.
\item 
\end{enumerate}



% \section{Вибір технологій для розроблення серверної частини}

% test

% \section{Вибір технологій для розроблення нейронної мережі}

% test

% \section{Вибір технологій для розроблення клієнтської частини}