% ********** Приклад оформлення пояснювальної записки **********
% *********  до атестаційної роботи ступеня бакалавра **********
% ************* автори: Тавров Д. Ю., Сахаров С. Ю. ************

% зазначаємо стильовий файл, який будемо використовувати
\documentclass{bachelor_thesis}

% \usepackage{fontspec}

\usepackage{verbatim}
\newcommand{\Mod}[1]{\ (\text{mod}\ #1)}
\DeclareMathOperator{\ind}{ind}


% \theoremstyle{Def}
% \newtheorem{Def}{Визначення}[chapter]

% починаємо верстку документа
\begin{document}

% створюємо титульний аркуш
% за допомогою спеціальної команди
% \maketitlepage{params},
% де params --- це розділені комами пари "параметр={значення}"
% \maketitlepage{
% % StudentName --- ПІБ студента
% 	StudentName={Рибак Богдан Сергійович},
% % StudentMale --- стать студента (true, якщо чоловік, false --- якщо жінка)
% 	StudentMale=true,
% % StudentGroup --- група студента
% 	StudentGroup={ФІ-23},
% % ThesisTitle --- тема роботи (без лапок)
% 	ThesisTitle={Методи реалізації алгоритмів дискретного логарифмування у скінченних полях вигляду $GF(p)$ для різних моделей обчислень},
% % Advisor --- посада, науковий ступінь, вчене звання, прізвище, ініціали керівника роботи
% 	Advisor={доцент, канд. техн. наук, доцент Петренко~П.~П.},
% % Reviewer --- посада, науковий ступінь, вчене звання, прізвище, ініціали рецензента
% 	Reviewer={проф., д-р техн. наук, проф. Сидоренко~С.~С.},
% % ConsultantChapter --- назва розділу, який консультував консультант
% % якщо консультантів роботи немає, цей параметр потрібно пропустити
% 	ConsultantChapter={зі спеціальних питань},
% % Consultant --- посада, науковий ступінь, вчене звання, прізвище, ініціали консультанта
% % якщо консультантів роботи немає, цей параметр потрібно пропустити
% 	Consultant={старший викладач, канд. техн. наук Бондаренко~Б.~Б.},
% % якщо в роботі є додатковий консультант, його можна описати аналогічно, використавши параметри:
% % ConsultantChapterSecond --- назва розділу, який консультував другий консультант
% % ConsultantSecond --- посада, науковий ступінь, вчене звання, прізвище, ініціали другого консультанта
% % якщо додаткового консультанта немає, відповідні параметри потрібно пропустити
% % ЗВЕРНІТЬ УВАГУ: консультант із нормоконтролю вставляється на титульний аркуш автоматично, і окремо прописувати його в команді \maketitlepage НЕПОТРІБНО
% % Year --- рік захисту роботи
% 	Year={2016}
% }


% створюємо завдання на дипломну роботу
% за допомогою спеціальної команди
% \assignment{params},
% % де params --- це розділені комами пари "параметр={значення}"
% \assignment{
% %	StudentName --- ПІБ студента в родовому відмінку
% 	StudentName={Іваненку Івану Івановичу},
% % StudentMale --- стать студента (true, якщо чоловік, false --- якщо жінка)
% 	StudentMale=true,
% %	ThesisTitle --- тема роботи (без лапок)
% 	ThesisTitle={Математичне та програмне забезпечення автоматизованої системи визначення емоційного стану людини за її зображенням},
% %	Advisor --- ПІБ, науковий ступінь, вчене звання керівника
% 	Advisor={Петренко Петро Петрович, канд. техн. наук, доцент},
% % Order --- дата та номер наказу
% 	Order={\invcommas{28}~травня~2016~р.~\No~000-C},
% % ApplicationDate --- 2. Термін подання студентом роботи
% 	ApplicationDate={\invcommas{15}~червня~2016~р.},
% %	InputData --- 3. Вихідні дані до роботи
% 	InputData={розроблювана система повинна працювати з нечіткими даними, мінімальна точність розпізнавання емоцій --- 70\%},
% %	Contents --- 4. Зміст роботи
% 	Contents={виконати аналіз існуючих методів розв'язання задачі, вибрати метод розпізнавання емоцій за зображенням людини, спроектувати автоматизовану систему розпізнавання емоцій, здійснити програмну реалізацію розробленої системи, провести тестування розробленої системи},
% %	Graphics --- 5. Перелік ілюстративного матеріалу
% 	Graphics={архітектурні графи нейронних мереж, блок-схеми розроблених алгоритмів, схема взаємодії модулів системи, знімки екранних форм},
% % ConsultantChapter --- назва розділу, який консультував консультант
% % якщо консультантів роботи немає, цей параметр потрібно пропустити
% 	ConsultantChapter={Розділ 3. Математичне забезпечення},
% % Consultant --- прізвише, ініціали та посада консультанта
% % якщо консультантів роботи немає, цей параметр потрібно пропустити
% 	Consultant={Бондаренко~Б.~Б., старший~викладач},
% % якщо в роботі є додатковий консультант, його можна описати аналогічно, використавши параметри:
% % ConsultantChapterSecond --- назва розділу, який консультував другий консультант
% % ConsultantSecond --- прізвише, ініціали та посада другого консультанта
% % якщо додаткового консультанта немає, відповідні параметри потрібно пропустити
% %	AssignmentDate --- дата видачі завдання
% 	AssignmentDate={\invcommas{15}~квітня~2016~р.},
% % Calendar --- внутрішня частина таблиці з календарним планом
% % кожний рядок таблиці повинен мати формат:
% % #1 & #2 & #3 & \\
% % де #1 --- номер з/п
% % #2 --- назва завдання
% % #3 --- термін виконання завдання
% % після кожного рядка, окрім останнього, потрібно на окремому рядку
% % залишати \hline
% 	Calendar={
% 	1 & Огляд літератури за тематикою та збір даних & 12.11.2015 & \\
% 	\hline
% 	2 & Проведення порівняльного аналізу математичних методів розпізнавання зображень & 14.12.2015 & \\
% 	\hline
% 	3 & Проведення порівняльного аналізу математичних методів розпізнавання емоційного стану за зображенням обличчя & 24.12.2015 & \\
% 	\hline
% 	4 & Підготовка матеріалів першого розділу роботи & 01.02.2016 & \\
% 	\hline
% 	5 & Розроблення математичного забезпечення для розпізнавання емоційного стану за статичним фронтальним зображенням обличчя & 01.03.2016 & \\
% 	\hline
% 	6 & Підготовка матеріалів другого розділу роботи & 15.03.2016 & \\
% 	\hline
% 	7 & Підготовка матеріалів третього розділу роботи & 05.04.2016 & \\
% 	\hline
% 	8 & Розроблення програмного забезпечення для розпізнавання емоційного стану за статичним фронтальним зображенням обличчя & 15.04.2016 & \\
% 	\hline
% 	9 & Підготовка матеріалів четвертого розділу роботи & 03.05.2016 & \\
% 	\hline
% 	10 & Оформлення пояснювальної записки & 01.06.2016 & \\
% 	},
% % StudentNameShort --- ініціали та прізвище студента
% 	StudentNameShort={Іваненко~І.~І.},
% %	AdvisorNameShort --- ініціали та прізвище керівника
% 	AdvisorNameShort={Петренко~П.~П.},
% % Year --- рік затвердження завдання
% 	Year={2015}
% }


% %!TEX root = ../abstract.tex
% створюємо анотацію
% \setcounter{page}{4}
\chapter*{Реферат}
\pagestyle{empty}
\setfontsize{14}
\thispagestyle{empty}
% далі пишемо текст анотації

% анотація повинна починатися інформацією про структуру роботи
% (кількість аркушів (БЕЗ ДОДАТКІВ!), додатків, посилань, рисунків і таблиць)
Роботу виконано на 42 аркушах, вона містить перелік посилань на використані джерела з $42$ найменувань.

% далі потрібно вказати мету роботи
\textbf{Метою} даної дипломної роботи ииии побудова програмного комплексу для розпізнавання промовленного тексту за відеорядом міміки обляччя людини.

\textbf{Об'єктом дослідження} є something.

\textbf{Предметом дослідження} є something.

Lorem ipsum dolor sit amet, consectetur adipiscing elit. Vivamus malesuada sapien mattis justo pellentesque commodo. 
Nulla aliquet lorem nec dolor pellentesque, rhoncus vestibulum velit auctor. Suspendisse potenti. Vivamus at sapien velit.
Maecenas rhoncus egestas purus sed cursus. Fusce posuere nisl quis sem laoreet faucibus. Ut cursus at libero et iaculis. 
Donec finibus, nunc sit amet cursus lobortis, turpis lectus consectetur neque, non accumsan felis diam eget lorem. Mauris 
auctor dolor arcu, a aliquam diam vestibulum ut. Maecenas a cursus lectus, egestas eleifend magna.
 
 % наприкінці анотації потрібно зазначити ключові слова
\MakeUppercase{Lorem, ipsum, dolor, sit, amet}

% створюємо анотацію англійською мовою
\chapter*{Abstract}
\thispagestyle{empty}
% анотація повинна починатися інформацією про структуру роботи
% (кількість аркушів (БЕЗ ДОДАТКІВ!), додатків, посилань, рисунків і таблиць)
The thesis is presented in N pages. It contains bibliography of N references.

% % далі потрібно вказати мету роботи

The \textbf{goal} Lorem ipsum dolor sit amet, consectetur

\textbf{The object} Lorem ipsum dolor sit amet, consectetur

\textbf{The subject} Lorem ipsum dolor sit amet, consectetur


% % далі потрібно вказати розглянуті методи та критерії, за яким вибрано один із них

Lorem ipsum dolor sit amet, consectetur adipiscing elit. Vivamus malesuada sapien mattis justo pellentesque commodo. 
Nulla aliquet lorem nec dolor pellentesque, rhoncus vestibulum velit auctor. Suspendisse potenti. Vivamus at sapien velit.
 Maecenas rhoncus egestas purus sed cursus. Fusce posuere nisl quis sem laoreet faucibus. Ut cursus at libero et iaculis. 
 Donec finibus, nunc sit amet cursus lobortis, turpis lectus consectetur neque, non accumsan felis diam eget lorem. Mauris 
 auctor dolor arcu, a aliquam diam vestibulum ut. Maecenas a cursus lectus, egestas eleifend magna.
 
 
% % далі потрібно коротко викласти суть роботи
% The selected  is implemented in parallel computation model and is executed on the selected  platform. From the results of this execution analysis of optimal algorithm parameters and approximate problem size solvable in the span of one calendar year were made.

% % далі потрібно подати відомості про апробацію роботи


% The results could be used for estimating attack cost on popular asymmetric .

% % наприкінці анотації потрібно зазначити ключові слова
\MakeUppercase{Lorem, ipsum, dolor, sit, amet}

% створюємо зміст

%!TEX root = ../abstract.tex
% створюємо анотацію
% \setcounter{page}{4}
\chapter*{Реферат}
\pagestyle{empty}
\setfontsize{14}
\thispagestyle{empty}
% далі пишемо текст анотації

% анотація повинна починатися інформацією про структуру роботи
% (кількість аркушів (БЕЗ ДОДАТКІВ!), додатків, посилань, рисунків і таблиць)
Роботу виконано на N аркушах, вона містить перелік посилань на використані джерела з $N^2$ найменувань.

% далі потрібно вказати мету роботи
\textbf{Метою} даної дипломної роботи є побудова something.

\textbf{Об'єктом дослідження} є something.

\textbf{Предметом дослідження} є something.

Lorem ipsum dolor sit amet, consectetur adipiscing elit. Vivamus malesuada sapien mattis justo pellentesque commodo. 
Nulla aliquet lorem nec dolor pellentesque, rhoncus vestibulum velit auctor. Suspendisse potenti. Vivamus at sapien velit.
 Maecenas rhoncus egestas purus sed cursus. Fusce posuere nisl quis sem laoreet faucibus. Ut cursus at libero et iaculis. 
 Donec finibus, nunc sit amet cursus lobortis, turpis lectus consectetur neque, non accumsan felis diam eget lorem. Mauris 
 auctor dolor arcu, a aliquam diam vestibulum ut. Maecenas a cursus lectus, egestas eleifend magna.
 
 % наприкінці анотації потрібно зазначити ключові слова
\MakeUppercase{Lorem, ipsum, dolor, sit, amet}

% створюємо анотацію англійською мовою
\chapter*{Abstract}
\thispagestyle{empty}
% анотація повинна починатися інформацією про структуру роботи
% (кількість аркушів (БЕЗ ДОДАТКІВ!), додатків, посилань, рисунків і таблиць)
The thesis is presented in N pages. It contains bibliography of N references.

% % далі потрібно вказати мету роботи

The \textbf{goal} Lorem ipsum dolor sit amet, consectetur

\textbf{The object} Lorem ipsum dolor sit amet, consectetur

\textbf{The subject} Lorem ipsum dolor sit amet, consectetur


% % далі потрібно вказати розглянуті методи та критерії, за яким вибрано один із них

Lorem ipsum dolor sit amet, consectetur adipiscing elit. Vivamus malesuada sapien mattis justo pellentesque commodo. 
Nulla aliquet lorem nec dolor pellentesque, rhoncus vestibulum velit auctor. Suspendisse potenti. Vivamus at sapien velit.
 Maecenas rhoncus egestas purus sed cursus. Fusce posuere nisl quis sem laoreet faucibus. Ut cursus at libero et iaculis. 
 Donec finibus, nunc sit amet cursus lobortis, turpis lectus consectetur neque, non accumsan felis diam eget lorem. Mauris 
 auctor dolor arcu, a aliquam diam vestibulum ut. Maecenas a cursus lectus, egestas eleifend magna.
 
 
% % далі потрібно коротко викласти суть роботи
% The selected  is implemented in parallel computation model and is executed on the selected  platform. From the results of this execution analysis of optimal algorithm parameters and approximate problem size solvable in the span of one calendar year were made.

% % далі потрібно подати відомості про апробацію роботи


% The results could be used for estimating attack cost on popular asymmetric .

% % наприкінці анотації потрібно зазначити ключові слова
\MakeUppercase{Lorem, ipsum, dolor, sit, amet}



% % створюємо додатки
% % перший додаток повинен містити лістинги розроблених програм
% \append{Лістинги програм}

% % кожний лістинг вставляється в додаток за допомогою спеціальної команди,
% % перший аргумент якої --- це заголовок, який з'являтиметься в тексті,
% % другий --- шлях до файлу з лістингом
% \listing{presenter.h --- прототип пред'явника-вчителя}{SRC/Presenter/presenter.h}

% \listing{presenter.cpp --- реалізація пред'явника-вчителя}{SRC/Presenter/presenter.cpp}

% \input{chapters/appendix_firstrun}

% % останній додаток повинен містити слайди презентації доповіді на захисті дипломної роботи
% \append{Ілюстративний матеріал}

% \begin{figure}[!htp]%
% 	\centering
% 	\includegraphics[scale = 0.43]{PNG/slide1.png}%
% 	\caption{Слайд 1}%
% 	\label{fig:p1}%
% \end{figure}

% \begin{figure}[!htp]%
% 	\centering
% 	\includegraphics[scale = 0.455]{PNG/slide2.png}%
% 	\caption{Слайд 2}%
% 	\label{fig:p2}%
% \end{figure}

\end{document}